% Fit the abstract on a single page
% \ESKDsetPadding{7mm}{4mm}

\newcommand{\abstracttile}[1]{%
\begingroup%
\centering\fontsize{18pt}{20pt}\selectfont\textbf{#1}\par%
\endgroup}

\thispagestyle{empty}
\ESKDthisStyle{empty}
% \linespread{1.05}\selectfont
\abstracttile{Анотація}
У даній бакалаврській роботі розглянуті питання виявлення аномалій в поведінці абонентів телефонної мережі. Досліджені різні способи виявлення аномалій. Розроблений спосіб виявлення аномалій на базі моделі Денінга, запропоновано модифікацію способу для врахування групових змін в поведінці абонентів.

Розроблена програма для виявлення аномалій абонентів телефонної мережі, досліджено роботу алгоритму у режимах без модифікації та з модифікацією.

\abstracttile{Аннотация}
В данной бакалаврской работе рассмотрены вопросы обнаружения аномалий в поведении абонентов телефонной сети. Исследованы различные способы выявления аномалий. Разработан способ обнаружения аномалий на базе модели Деннинга, предложена модификация способа для учета групповых изменений в поведении абонентов.

Разработана программа для обнаружения аномалий в поведении абонентов телефонной сети, исследована работа алгоритма в режимах без модификации и с модификацией.

\abstracttile{Abstract}
In this work the problem of detection anomalies in the behavior of telephone subscribers is studied. Various methods of anomaly detection are analyzed. The method of anomaly detection is developed, approach of improving method to take into consideration group behavior changes is proposed.

The program for detecting anomalies in subscriber behavior is developed, results of simulation in two modes, with and without modification, are given.
