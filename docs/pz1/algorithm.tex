\newpage
\bignumberedpart{Розробка програмного продукту}
\subsection{Синтез алгоритму}

За моделлю Денінга \cite{denning1987intrusion}, компонентами є  % 2 сторінка, овервью

\begin{ESKDexplanation}
\item суб'єкт системи --- абонент, або його телефонний номер;
\item об'єкт системи --- також абонент, друга сторона виклику;
\item записи журналу --- записи, згенеровані внаслідок дій суб'єктів над 
      об'єктами, тобто згенеровані АТС записи CDR;
\item профілі --- структури, згенеровані системою у відповідності до дій 
      суб'єктів над об'єктами, тобто на базі журнальних записів;
\item аномальні записи --- записи CDR, які позначені системою як нехарактерні
      для суб'єкту.
\end{ESKDexplanation}

Виявлення шахрайства в реальному часі базується на гіпотезі, що така поведінка 
включає аномальне використання системи. Тому, шахрайські дії можуть бути 
виявлені як аномальнії у використанні системи. Приклади:
% \cite{denning1987intrusion}.

\begin{itemize}
  \item Клонування телефонів - дані дзвінків перехрещуються у часі, кількість 
        зростає.
  \item Шахрайство при підключенні - до системи підключається новий абонент, 
        шаблон якого ще не побудований (та не зареєстрований в системі). 
        Система дозволяє виявити таких порушників ще до першого зняття коштів з
         рахунку.
  \item Крадіжка телефону - змінюється шаблон поведінки користувача,
  \item Злам АТС - різке збільшення трафіку,
\end{itemize}

Спосіб визначення аномалій за моделлю Деннінга \cite{denning1987intrusion} можна модифікувати так, що змінна 
спостереження - не дискретна величина кількості подій за 
фіксований проміжок часу, а миттєва частота таких подій. Таким 
чином, ми поєднуємо метод числових рядів та статистичного методу
з використанням перших двох моментів розподілу - середнього 
значення та середньоквадратичного відхилення.

Спосіб агрегації даних модифікований так, що одночасно і вцілому і кожен абонент окремо.\TBD % в виводи

Тобто, маючи ${T_i^p}$ можна 
розрахувати в 
реальному часі інтенсивність дзвінків\TBD


Метод полягає в побудові шаблону поведінки користувача на основі декількох тижнів спостереження, і згодом виявлення дзвінків, які виходять за рамки поточного шаблону. Відповідно, робота алгоритму ділиться на 2 етапи: режим навчання і робочий режим.

Враховуючи випадкову природу здійснення телефонного дзвінка по відношенню до оператора, для аналізу поведінки можна виходити з припущення, що кількість дзвінків за певний проміжок часу буде розподілено по розподілу Пуассона.

Також врахуємо, що поведінка користувача залежить від дня тижня. Шаблон поведінки користувача - функція, що показує інтенсивність дзвінків від часу
\begin{equation}\label{eq:pattern_a}P: t \rightarrow \lambda \end{equation}

\begin{ESKDexplanation}
  \item де $\lambda$ - це кількість дзвінків на одиницю часу.
\end{ESKDexplanation}

Тоді запис поведінки користувача можна визначити як вектор довжиною $L = H * 7$, де $H$ - число розбиття доби, а 7 - кількість днів у тижні. Для зменшення випадкової складової, складається декілька записів поведінки по тижнях. Кількість збережених записів $W$ впливає на точність кінцевого шаблону поведінки (\ref{eq:pattern_a})

\begin{equation}\label{eq:pattern_a2}P = (\overline{\lambda_1}, \overline{\lambda_2}, ..., \overline{\lambda_L}) \end{equation}

який можна визначити як усереднений вектор записів поведінки за попередні W тижнів, де середнє значення рахується як експоненціально зважене рухоме середнє з вікном в кількість записів:

\begin{equation}\label{eq:ema}EMA_n = (1-\alpha) \cdot x_n + \alpha \cdot EMA_{n-1} \end{equation}

що можна призвести до не-рекурсивної формули (\ref{eq:ema_nonrecursive}):

\begin{equation}\label{eq:ema_nonrecursive}EMA = \frac{{x}_{n} + \alpha {x}_{n -1} + {\alpha} ^ {2} {x}_{n -2} +...+ {\alpha} ^ {n -1} {x}_{1}}{1+ \alpha + {\alpha} ^ {2} +...+ {\alpha} ^ {n -1}} \end{equation}

Це дозволяє зменшити запізнювання, надаючи більшого значення останнім значенням.

\begin{equation}\label{eq:ema_nonrecursive}\overline{{\lambda}_{i}} = \frac{{\lambda}_{i w} + \alpha {\lambda}_{i(w -1)} +...+ {\alpha} ^ {n -1} {\lambda}_{i1}}{1+ \alpha + {\alpha} ^ {2} +...+ {\alpha} ^ {w -1}} \end{equation}

В режиме обучения система принимает  записи о звонках, измеряет частоту звонков и фиксирует её в записях поведения. Когда нужное количество записей сохранено (в зависимости от выбранного критерия, или достигается заданная дисперсия, или задается необходимое количество записей), система переводится в рабочий режим для конкретного абонента. То есть в один момент времени часть абонентов может обрабатываться в режиме обучения, а часть — в рабочем режиме. Это необходимо, так как во время работы системы могут подключиться новые абоненты.
Подсчёт текущей частоты  делается на основе времени инициации последних K звонков для каждого абонента отдельно. Имея вектор  отметок времени инициации звонков (в секундах) рассчитываем предполагаемую частоту за определённый промежуток времени T: 

(5)

Предполагаемая частота за время одной ячейки шаблона:

(6)

где  – текущая частота за секунду

(7)

 - среднее время между звонками. Для уменьшения эффекта запаздывания за изменением частоты, здесь также целесообразно использовать  экспоненциально взвешенное среднее значение:

(8)

Тогда:

(9)

В рабочем режиме система продолжает фиксировать записи поведения, то есть обучение не останавливается. В этом режиме начинает работать алгоритм кластеризации, который классифицирует шаблоны пользователей на k кластеров. Количество классов может быть варьировано с измерением для обеспечения точного разделения абонентов по характеру использования системы. Для кластеризации можно использовать алгоритм k-means, запускаемый периодически после записи очередной записи поведения (раз в неделю), но в виду его сложности для большого количества абонентов целесообразно использовать его потоковую модификацию [6], что позволит классифицировать шаблоны сразу после получения новых данных.
Помимо этого включается проверка каждого звонка на соответствие шаблону поведения. Проверка на соответствие может осуществляться как с использованием доверительных интервалов, так и проверкой с учетом дисперсии. Пусть предполагаемый доверительный интервал   с необходимой надёжностью, задаваемой оператором, а отклонение от предполагаемого значения для интенсивности рассматриваемого временного промежутка  из шаблона поведения P: 

(10)

где i – номер ячейки шаблона. Тогда при  значение текущей частоты находится в пределах ожидаемой.
Для установившегося состояния системы, среднее значение отклонений:

(11)

где i – номера абонентов класса C, будет около нуля. Если же будет происходить сезонное изменение или некоторый иной фактор, который влияет на характер использования системы класса или нескольких классов пользователей, тренд покажет характер этих изменений.
Смыслом функции тренда является процент отклонения класса абонентов от предыдущего характера использования системы. Поэтому, для уменьшения ложных срабатываний системы обнаружения аномального поведения, необходимо расширить доверительный интервал на вычисленный тренд.
Таким образом, интервенция может быть обнаружена сравнением текущей частоты с границами доверительного интервала, который равен:

(12)

Собирая все части воедино, получаем алгоритм:

\begin{algorithm}[H]
\KwData{Неперервний потік CDR записів}
\KwResult{Сигнали про інтервенції}
\SetAlgoLined

  \While{є нові записи CDR} {
    \If{(шаблон в робочому режимі) и (CDR не відповідає шаблону)} {
      сигнал про інтервенцію\;
    }
   модифікувати шаблон поведінки\;
   перерахувати миттєву інтенсивність\;
   перерахувати тренд\;
   \If{перерахувати тренд $T_{cluster}$} {
    ініціювати кластеризацію абонентів\;
   }
  }

\end{algorithm}
Где Tcluster – время с последней кластеризации. 
Последний пункт может быть модифицирован для потоковой кластеризации.


    \subsection{Алгоритм}

  Шаблон поведінки - вектор
  $P = (\overline{\lambda_1}, \overline{\lambda_2}, \dots, \overline{\lambda_L})$,
  який визначається як 

Задачею є згенерувати подібний потік за шаблонами $P_i$, $i \in \overline{1..L} $

	\subsection{Модифікація алгоритму для роботи в паралельній комп'ютерній системі}

	Кластерізація та розрахунок шаблонів поведінки це важкі обчислення, які не потребують обробки в реальному часі але потребують обробки великого обсягу журналу, тому їх можна винести у окрему періодичну задачу. Такі задачі добре розв'язуються за допомогою парадігми Map-Reduce.

	Все інше - в реальному часі \TBD.

\subsection{Архітектура системи}

\subsubsection{Журнал дзвінків}
  Вхідний потік подій ${T_i^p}$ складається із записів даних виклику, модель 
  даних записів - CDR - Call Detail Record. В кожен запис входять такі поля (в 
  дужках позначені назви комірок у системі АТС Asterisk):

   \begin{itemize}
    \item номер абонента, що ініціював дзвінок (src);
    \item номер абонента, кому був адресований дзвінок (dst);
    \item час ініціювання дзвінка (start);
    \item час відповіді на дзвінок (answer);
    \item час завершення дзвінку (end);
    \item довжина дзвінку -- різниця між полями end та start (duration);
    \item довжина розмови -- різниця між полями end та answer (billsec);
    \item статус дзвінка - прийнятий, відхилений, зкинутий, інше (disposition).
  \end{itemize}

  Статус дзвінка може приймати значення

  \begin{itemize}
    \item NO ANSWER --- адресат не прийняв дзвінок;
    \item FAILED --- невідомий адресат або інша помилка системи/ліній зв'язку;
    \item BUSY --- лінія адресата зайнята / адресат скинув дзвінок;
    \item ANSWERED --- успішний дзвінок;
    \item UNKNOWN --- статус невідомий.
  \end{itemize}

  може бути розширений в залежності від використовуваних інструментів та аппаратного/програмного забеспечення АТС.

  \begin{table}[h]
  \footnotesize
  \caption{Приклад журналу CDR}
        \begin{tabularx}{\textwidth}{| X | X | X | X | X | X | X | X |}
          \hline
          Абонент що дзвонить & Абонент що викликається & Ініц. дзвінка (с) & Від-повідь (с) & Кінець (с) & Три-валість (с) & Три-валість розмови (с) & Статус \\ \hline
          \scriptsize{0000000244} & \scriptsize{0007679961} & 365996 & 366049 & 366095 & 99 & 46 & \scriptsize{ANSWERED} \\ \hline
          \scriptsize{0000000238} & \scriptsize{0000434356} & 376215 & 376230 & 376354 & 139 & 124 & \scriptsize{ANSWERED}  \\ \hline
      \end{tabularx}
      \label{tab:cdr-log-example}
  \end{table}

\subsubsection{\TBD}

EMA для адаптивності


\subsection{Вибір інструментів для реалізації системи}
    \TBD

\subsection{Структура розробленої програми}
    \TBD


\newpage
\subsection*{Висновки до розділу 3}
\addcontentsline{toc}{subsection}{Висновки до розділу 3}
    \TBD
