\newpage
\bignumberedpart{Алгоритм TBD}
    \subsection{Алгоритм}

  Шаблон поведінки - вектор $P = (\overline{\lambda_1}, \overline{\lambda_2}, \dots, \overline{\lambda_L})$, який визначається як 

Задачею є згенерувати подібний потік за шаблонами $P_i$, $i \in \overline{1..L} $

	\subsection{Оптимізація алгоритму в паралельній комп'ютерній системі}
	Кластерізація та розрахунок шаблонів поведінки це важкі обчислення, які не потребують обробки в реальному часі але потребують обробки великого обсягу журналу, тому їх можна винести у окрему періодичну задачу. Такі задачі добре розв'язуються за допомогою парадігми Map-Reduce.

	Все інше - в реальному часі \TBD.


\newpage
\subsection*{Висновки}
\addcontentsline{toc}{subsection}{Висновки}
    \TBD
