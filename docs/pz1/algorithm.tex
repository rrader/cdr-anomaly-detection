\newpage
\bignumberedpart{Алгоритм TBD}
\subsection{Синтез алгоритму}

За \cite{denning1987intrusion}  % 2 сторінка, овервью

\begin{ESCDdescription}
\item суб'єкт системи --- абонент, або його телефонний номер;
\item об'єкт системи --- телефонна лінія (включаючи АТС та апарати);
\item записи журналу --- записи, згенеровані внаслідок дій суб'єктів над об'єктами, тобто згенеровані АТС записи CDR;
\item профілі --- структури, згенеровані системою у відповідності до дій суб'єктів над об'єктами, тобто на базі журнальних записів;
\item аномальні записи --- записи CDR, які позначені системою як нехарактерні.
\end{ESCDdescription}

Виявлення шахрайства в реальному часі базується на гіпотезі, що така поведінка включає аномальне використання системи. Тому, шахрайські дії можуть бути виявлені як аномальнії у використанні системи. Приклади: % \cite{denning1987intrusion}.

\begin{itemize}
  \item Клонування телефонів - дані дзвінків перехрещуються у часі, кількість зростає.
  \item Шахрайство при підключенні - до системи підключається новий абонент, шаблон якого ще не побудований (та не зареєстрований в системі). Система дозволяє виявити таких порушників ще до першого зняття коштів з рахунку.
  \item Крадіжка телефону - змінюється шаблон поведінки користувача,
  \item Злам АТС - різке збільшення трафіку,
\end{itemize}


    \subsection{Алгоритм}

  Шаблон поведінки - вектор $P = (\overline{\lambda_1}, \overline{\lambda_2}, \dots, \overline{\lambda_L})$, який визначається як 

Задачею є згенерувати подібний потік за шаблонами $P_i$, $i \in \overline{1..L} $

	\subsection{Оптимізація алгоритму в паралельній комп'ютерній системі}
	Кластерізація та розрахунок шаблонів поведінки це важкі обчислення, які не потребують обробки в реальному часі але потребують обробки великого обсягу журналу, тому їх можна винести у окрему періодичну задачу. Такі задачі добре розв'язуються за допомогою парадігми Map-Reduce.

	Все інше - в реальному часі \TBD.

\subsection{Журнал дзвінків}
  Вхідний потік подій ${T_i^p}$ складається із записів даних виклику, модель даних записів - CDR - Call Detail Record. В кожен запис входять такі поля (в дужках позначені назви комірок у системі АТС Asterisk):

   \begin{itemize}
    \item номер абонента, що ініціював дзвінок (src);
    \item номер абонента, кому був адресований дзвінок (dst);
    \item час ініціювання дзвінка (start);
    \item час відповіді на дзвінок (answer);
    \item час завершення дзвінку (end);
    \item довжина дзвінку -- різниця між полями end та start (duration);
    \item довжина розмови -- різниця між полями end та answer (billsec);
    \item статус дзвінка - прийнятий, відхилений, зкинутий, інше (disposition).
  \end{itemize}

  Статус дзвінка може приймати значення

  \begin{itemize}
    \item NO ANSWER --- адресат не прийняв дзвінок;
    \item FAILED --- невідомий адресат або інша помилка системи/ліній зв'язку;
    \item BUSY --- лінія адресата зайнята / адресат скинув дзвінок;
	\item ANSWERED --- успішний дзвінок;
	\item UNKNOWN --- статус невідомий.
  \end{itemize}

  може бути розширений в залежності від використовуваних інструментів та аппаратного/програмного забеспечення АТС.

  \begin{table}[h]
  \footnotesize
  \caption{Приклад журналу CDR}
        \begin{tabularx}{\textwidth}{| X | X | X | X | X | X | X | X |}
          \hline
          Абонент що дзвонить & Абонент що викликається & Ініц. дзвінка (с) & Від-повідь (с) & Кінець (с) & Три-валість (с) & Три-валість розмови (с) & Статус \\ \hline
          \scriptsize{+380000000244} & \scriptsize{+380007679961} & 365996 & 366049 & 366095 & 99 & 46 & \scriptsize{ANSWERED} \\ \hline
          \scriptsize{+380000000238} & \scriptsize{+380000434356} & 376215 & 376230 & 376354 & 139 & 124 & \scriptsize{ANSWERED}  \\ \hline
      \end{tabularx}
      \label{tab:cdr-log-example}
  \end{table}

\newpage
\subsection*{Висновки}
\addcontentsline{toc}{subsection}{Висновки}
    \TBD
