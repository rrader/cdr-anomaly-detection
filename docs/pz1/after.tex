% \section*{Висновки}
% \addcontentsline{toc}{section}{Висновки}
\newpage
\bigpart{Висновки}

Проблема виявлення аномальної поведінки стоїть дуже гостро, її важливість зростає з кожним роком. Оператори телефонного зв'язку розроблюють та встановлюють системи виявлення аномалій та вторгнень. Такі системи дозволяють зменшити кількість шахраїв, зменшити витрати на відшкодування збитків та покращити якість послуг, що надаються.

Модель Денінга є класичною роботою, в якій описані можливі способи побудови системи виявлення вторгнень (ADS). Ця робота лежить в основі майже всіх існуючих зараз ADS. Разом з тим, виявлення аномалій у телефонній мережі має свою специфіку в порівнянні з комп'ютерними мережами, тим не менш ідеї, описані в моделі успішно застосовуються і для телефонних мереж також. Модель розглядає користувачів як непов'язаних сутностей та не враховує змін групових поведінки, причиною якої є деяка зовнішня причина.

В більшості випадків такі системи не є автономними, та змушують операторів опрацьовувати в ручному режимі потенційних шахраїв. Не врахування групових змін в поведінці може додати дуже велику кількість хибнопозитивних сигналів про інтервенцію в поведінку користувачів.

Розроблений спосіб виявлення аномалій з врахуванням групової поведінки дає можливість суттєво зменшити кількість хибнопозитивних сигналів. Написана програма проаналізована на тестових даних, показано, що врахування групової зміни поведінки не впливає на виявлення одиночних інтервенцій, але разом з тим дійсно не визначає зміни в поведінці як інтервенції, якщо вони зроблені одночасно групою користувачів.
