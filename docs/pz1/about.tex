\newpage
\bignumberedpart{Постановка задачі виявлення аномалій у поведінці користувачів}

\subsection{Передумови}
	Розглянемо офісну міні-АТС (PBX), яка обслуговує телефонних абонентів одного або кількох офісних приміщень або будинків. В результаті використання ресурсів неавторизованим користувачем, або авторизованим нелегально, можуть бути вагомі неотримані прибутки компанією оператором. Ці неотримані прибутки можуть бути покладені на абонентом (в даному випадку абонентом є компанія-замовник АТС, де кінцевими абонентами є співробітники компанії), АТС якого був використаний нелегально. В разі доведення абонентом (можливо в судовому порядку), що втрати були понесені внаслідок атаки, оператор може не отримати ці кошти, хоча послуга вже отримана та зроблені витрати на надання послуги.

	Розглянемо телефонну мережу оператора стаціонарного чи мобильного зв'язку. В такій мережі існує велика кількість АТС (PBX), розподілених по території та які обслуговують велику кількість абонентів. Внаслідок нелегального використання послуг оператор може понести великі операційні збитки, а також істотні збитки у судових справах.

\subsection{Основні поняття}
	Наразі існують два доступних види телефонних комунікацій. Перший це наземні аналогові лінії, які передають неперервний хвилевий сигнал, другий використовує цифрову передачу сигналу, в якому трафік кодується за допомогою швидкої передачі бінарних імпульсів. На даний момент оператори мобільного зв'язку використовують цифровий спосіб передаці голосових даних та більшість операторів наземного перейшли на цифрові системи. 

	Протокол SIP у телефонії використовується майже скрізь, цифрові системи АТС (PBX) характеризуються легкістю аналізу даних. Call Detail Record (CDR), також відомий як запис даних виклику, є записом у журналі телефонної станції або іншого телекомунікаційного обладнання, що містить атрибути, характерні для одного телефонного дзвінка або іншої послуги зв'язку, яка була оброблена системою.

	Види шахрайства та втручання в систему були досліджені у роботі \cite{barson1996detection} та компанією TransNexus \cite{transnexus2012voipfraud},

\begin{itemize}
  \item Клонування телефонів - отримання доступу до мережі шляхом емуляції ідентифікаційного коду
другого справжнього мобільного телефону. Дані, що містяться на чіпі мобільного телефону (або SIM-картці) копіюються з одного мобільного телефону на інший. Цей тип шахрайства можна виявити в пересіченні телефонних дзвінків у часі,
  \item Шахрайство при підключенні - використання фальшивих ідентифікаційних даних для підключення до мережі. Цей тип шахрайства зазвичай не може бути визначений до першого зняття грошей з рахунку. Дані підключення також можуть бути скопійовані, тоді у мережі будуть присутні кілька телефонів, підключених з одними ідентифікаційними даними,
  \item Крадіжка телефону - легитимний власник телефону втрачає можливість робити дзвінки, а витрати робить неавторизована особа,
  \item Злам АТС - зазвичай використовується для виконання міжнародних дзвінків через чужу телефонну станцію. шахраї можуть використовувати вразливості у ПЗ АТС та можуть згенераувати значну кількість трафіку,
  \item Дзвінки на нерозподілені номери - шахраї можуть штучно створювати неіснуючі напрямки дзвінків та реєструватися єдиними провайдерами, що можуть з'єднати з номерами цих напрямків. Створюючи трафік на такі неіснуючі номери, дзвінки будуть перенаправлятися на компанію-шахрая.
\end{itemize}

Основними моментами для визначення втручання у систему можна визначити \cite{barson1996detection}:

\begin{itemize}
  \item Шахрайство є динамічним за своєю природою: нечесна поведінка буде змінюватися з плином часу,
  \item Розмірність задачі є достатньо великою за рахунок кількості абонентів, що мають бути відслідковані одночасно,
  \item Швидке виявлення шахрайства необхідно: збитки від шахрайства, як правило, ростуть експоненційно \cite{bliss1993fraud},
  \item Система має бути прозорою, клієнт не повинен бачити систему виявлення шахрайства в дії. Система виявлення аномалій не повинна бути максимально розумною, щоб надавати найменшу кількість хибно-позитивних та хибно-негативних сигналыв, але і не має бути автоматичною. Телефон не має бути заблокований, якщо немає достатньої впевненості, що це є зловмисник.
\end{itemize}

\subsection{Формальна постановка задачі}


	Розглянемо потік подій $T$, де кожна подія $T_i$ характеризується кортежем $(src, time)$, де
  \begin{description}
    \item[src] - номер абонента, що ініціював дзвінок,
    \item[time] - час ініціювання дзвінка.
  \end{description}

  Кожен абонент має унікальний номер $src$, за яким його можна ідентифікувати. Задача - визначити у реальному часі чи є потік подій для конкретного абонента аномальним, чи співпадає із очікованим шаблоном поведінки.

\subsection{Журнал дзвінків}
\TBD не в этом разделе

  Вхідний потік подій складається із записів даних виклику (CDR - Call Detail Record). 

  \begin{table}[h]
  \footnotesize
  \caption{Приклад журналу CDR}
        \begin{tabularx}{\textwidth}{| X | X | X | X | X | X | X | X |}
          \hline
          Абонент що дзвонить & Абонент що викликається & Ініц. дзвінка (с) & Від-повідь (с) & Кінець (с) & Три-валість (с) & Три-валість розмови (с) & Статус \\ \hline
          \scriptsize{+380000000244} & \scriptsize{+380007679961} & 365996 & 366049 & 366095 & 99 & 46 & \scriptsize{ANSWERED} \\ \hline
          \scriptsize{+380000000238} & \scriptsize{+380000434356} & 376215 & 376230 & 376354 & 139 & 124 & \scriptsize{ANSWERED}  \\ \hline
      \end{tabularx}
      \label{tab:cdr-log-example}
  \end{table}


\subsection{\TBD}
   \TBD
   \TBD
   \TBD

\newpage
\subsection*{Висновки}
\addcontentsline{toc}{subsection}{Висновки}
    \TBD


