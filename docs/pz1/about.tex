\newpage
\bignumberedpart{Постановка задачі виявлення аномалій у поведінці користувачів}

\subsection{Опис предметної області}
  Останнім часом в світі переконались, що навіть найнадійніші системи захисту не здатні захистити від атак комп'ютерні системи державних і комерційних установ. Одна з причин - у тому, що в більшості систем безпеки застосовують стандартні механізми захисту: ідентифікацію та аутентифікацію, механізми обмеження доступу до інформації згідно з правами суб'єкта і криптографічні механізми. Такий традиційний підхід має певні недоліки, а саме: незахищеність від власних користувачів - зловмисників, розмитість поділу суб'єктів системи на "своїх" і "чужих" через глобалізацію інформаційних ресурсів, порівняна легкість підбору паролів внаслідок використання їхнього змістового різновиду, зниження продуктивності й ускладнення інформаційних комунікацій внаслідок обмеження доступу до ресурсів організації. Тому виникла потреба в механізмах, які би доповнювали традиційні та давали можливість виявити спроби несанкціонованого доступу й інформували про це відповідальних за безпеку або реагували у відповідь. Важливим фактором є те, щоб такі системи могли протистояти атакам, навіть якщо зловмисник вже був аутентифікований та авторизований і з формального погляду додержання прав доступу мав необхідні повноваження на свої дії. Такі функції і виконують системи виявлення вторгнень IDS (intrusion detection systems). Оскільки передбачити всі сценарії розгортання подій в системі з активним "чужим"  
  суб'єктом неможливо, потрібно або якомога детальніше описати можливі "зловмисні" сценарії або ж, навпаки, - "нормальні" і прийняти, що всяка активність, на яку не поширюється прийняте розуміння "нормальності", є небезпечною. Системи IDS поділяються на системи, що реагують на відомі атаки, - системи виявлення зловживань MDS (misuse detection systems) і системи виявлення аномалій ADS (anomaly detection systems), які реєструють відхилення розвитку системи від нормального перебігу. \TBD

\subsection{Приклади вторгнення}

	Розглянемо офісну міні-АТС (PBX), яка обслуговує телефонних абонентів одного або кількох офісних приміщень або будинків. В результаті використання ресурсів неавторизованим користувачем, або авторизованим нелегально, можуть бути вагомі неотримані прибутки компанією оператором. Ці неотримані прибутки можуть бути покладені на абонентом (в даному випадку абонентом є компанія-замовник АТС, де кінцевими абонентами є співробітники компанії), АТС якого був використаний нелегально. В разі доведення абонентом (можливо в судовому порядку), що втрати були понесені внаслідок атаки, оператор може не отримати ці кошти, хоча послуга вже отримана та зроблені витрати на надання послуги.

	Розглянемо телефонну мережу оператора стаціонарного чи мобильного зв'язку. В такій мережі існує велика кількість АТС (PBX), розподілених по території та які обслуговують велику кількість абонентів. Внаслідок нелегального використання послуг оператор може понести великі операційні збитки, а також істотні збитки у судових справах.

\subsection{Основні поняття}

  Система виявлення вторгнень - програмний або апаратний засіб, призначений для виявлення фактів несанкціонованого доступу в мережу або несанкціонованого управління ними в основному через Інтернет. Відповідний англійський термін - Intrusion Detection System (IDS).

  Для запобігання шахрайству у мережах використовують мережеві екрани, що можуть блокувати або дозволяти певний трафік через себе.
  Основна різниця між мережевим екраном (файрволом) та системою виявлення вторгнення полягає у області дії. Мережевий екран діє на границі мережі, тоді як система виявлення вторгнень - всередені, вона пропускає через себе весь трафік та виявляє аномальні ділянки.

	Наразі існують два доступних види телефонних комунікацій. Перший це наземні аналогові лінії, які передають неперервний хвилевий сигнал, другий використовує цифрову передачу сигналу, в якому трафік кодується за допомогою швидкої передачі бінарних імпульсів. На даний момент оператори мобільного зв'язку використовують цифровий спосіб передаці голосових даних та більшість операторів наземного перейшли на цифрові системи.

	Протокол SIP у телефонії використовується майже скрізь, цифрові системи АТС (PBX) характеризуються легкістю аналізу даних. Call Detail Record (CDR), також відомий як запис даних виклику, є записом у журналі телефонної станції або іншого телекомунікаційного обладнання, що містить атрибути, характерні для одного телефонного дзвінка або іншої послуги зв'язку, яка була оброблена системою.

	Види шахрайства та втручання в систему були досліджені у роботі \cite{barson1996detection} та компанією TransNexus \cite{transnexus2012voipfraud},

\begin{itemize}
  \item Клонування телефонів - отримання доступу до мережі шляхом емуляції ідентифікаційного коду
другого справжнього мобільного телефону. Дані, що містяться на чіпі мобільного телефону (або SIM-картці) копіюються з одного мобільного телефону на інший. Цей тип шахрайства можна виявити в пересіченні телефонних дзвінків у часі,
  \item Шахрайство при підключенні - використання фальшивих ідентифікаційних даних для підключення до мережі. Цей тип шахрайства зазвичай не може бути визначений до першого зняття грошей з рахунку. Дані підключення також можуть бути скопійовані, тоді у мережі будуть присутні кілька телефонів, підключених з одними ідентифікаційними даними,
  \item Крадіжка телефону - легитимний власник телефону втрачає можливість робити дзвінки, а витрати робить неавторизована особа,
  \item Злам АТС - зазвичай використовується для виконання міжнародних дзвінків через чужу телефонну станцію. шахраї можуть використовувати вразливості у ПЗ АТС та можуть згенераувати значну кількість трафіку,
  \item Дзвінки на нерозподілені номери - шахраї можуть штучно створювати неіснуючі напрямки дзвінків та реєструватися єдиними провайдерами, що можуть з'єднати з номерами цих напрямків. Створюючи трафік на такі неіснуючі номери, дзвінки будуть перенаправлятися на компанію-шахрая.
\end{itemize}

Основними моментами для визначення втручання у систему можна визначити \cite{barson1996detection}:

\begin{itemize}
  \item Шахрайство є динамічним за своєю природою: нечесна поведінка буде змінюватися з плином часу,
  \item Розмірність задачі є достатньо великою за рахунок кількості абонентів, що мають бути відслідковані одночасно,
  \item Швидке виявлення шахрайства необхідно: збитки від шахрайства, як правило, ростуть експоненційно \cite{bliss1993fraud},
  \item Система має бути прозорою, клієнт не повинен бачити систему виявлення шахрайства в дії. Система виявлення аномалій не повинна бути максимально розумною, щоб надавати найменшу кількість хибно-позитивних та хибно-негативних сигналыв, але і не має бути автоматичною. Телефон не має бути заблокований, якщо немає достатньої впевненості, що це є зловмисник.
\end{itemize}

\subsection{Формальна постановка задачі}


	Розглянемо потік подій $\{T_i\}, i = \overline{1..n}$, де кожна подія $T_i$ характеризується кортежем
  \begin{equation}\label{eq:tuple} (src, time) \end{equation}
  \begin{ESKDexplanation}
    \item де src -- номер абонента, що ініціював дзвінок;
    \item time -- час ініціювання дзвінка.
  \end{ESKDexplanation}

  Кожен абонент має унікальний номер $src$, за яким його можна ідентифікувати. Задача розроблюваного алгоритму та системи - визначити у реальному часі чи є потік подій для конкретного абонента аномальним, чи співпадає із очікованим шаблоном поведінки.

  \begin{equation}\label{eq:stream}\{T_i^{p}\} = \{x | x \in T_i \wedge x_1 = p\}\end{equation}
  \begin{ESKDexplanation}
    \item де p - номер телефону.
  \end{ESKDexplanation}

  Тоді потік можна записати як сукупність потоків 

  \begin{equation}\label{eq:stream_sum}\{T_i\} = \{T_i^{p_1}\} \cup \{T_i^{p_2}\} \cup ... \cup \{T_i^{p_k}\} \end{equation}
  \begin{ESKDexplanation}
    \item де k - кількість телефонних номерів.
  \end{ESKDexplanation}

  Шаблон поведінки користувача - функція, що показує інтенсивність дзвінків від часу
  \begin{equation}\label{eq:pattern}P: t \rightarrow \lambda \end{equation}

  \begin{ESKDexplanation}
    \item де $\lambda$ - це кількість дзвінків на одиницю часу.
  \end{ESKDexplanation}

  Для кожного моменту часу $t$ можна визначити частоту, де шаблонна частота 
\begin{equation}\label{eq:pattern_freq} \lambda_t^p = P^p(t) \end{equation}

  а фактична частота
 \begin{equation}\label{eq:real_freq} \lambda_{rt}^p = F(\{T_i^p\}) \end{equation}
 \begin{ESKDexplanation}
    \item де $F$ - це функція розрахунку поточної частоти дзвінків.
  \end{ESKDexplanation}

  Задача 1: Сформувати шаблони поведінки для кожного номеру за історією подій: $\{T_i^{p}\} \rightarrow P^{p}$.

  Задача 2: Визначити функцію $F({T_i^p})$ (\ref{eq:real_freq}) На базі cформованих шаблонів поведінки $P^{p}$ та потоку подій $\{T_i\}, i = \overline{1..n}$ визначити чи є подія $T_{n+1}$ аномальною.

\newpage
\subsection*{Висновки до розділу 1}
\addcontentsline{toc}{subsection}{Висновки до розділу 1}
    \TBD


