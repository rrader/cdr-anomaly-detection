\newpage
\bignumberedpart{Огляд існуючих розв'язків}
\subsection{Стан проблеми \TBD}

\TBD

Відомо, що атака є багатокроковим процесом, його здійснення потребує високої кваліфікації зловмисника. Тому найпростішим способом злому або приведення системи в недієздатний стан є застосування "експлойтів", тобто вже написаних модулів, що реалізують необхідні етапи атаки. Величезна кількість експлойтів доступна через Internet, що привело до поширення саме такого виду атаки. Як наслідок, часто послідовність аудит-подій, що реалізують атаку, фіксована \cite{kolodchak2012modern}.

Робота MDS базується на створенні шаблонів таких атак. їх рівень абстракції може бути простим, таким як наявність певних значень у заголовку мережевого пакета або послідовності команд у файлі аудиту, або складним, таким як проходження траєкторії системи у просторі станів через певні небезпечні стани. Системи захисту такого типу ефективні, коли схема атак відома, однак у випадках невідомої атаки або відхилень перебігу атаки від шаблону виникають проблеми, тому слід підтримувати велику базу даних для кожної атаки та її варіації і організовувати безперервне поповнення баз шаблонів.

Основним припущенням ADS є те, що дії зловмисника (події в атакованій системі) обов'язково відрізняються від поведінки звичайного користувача (від подій в нормальному стані), тобто є аномаліями. Тому такі системи здатні реєструвати і невідомі атаки. Роботі ADS передує період накопичення інформації, коли складається концепція нормальної активності системи, процесу чи користувача. Вона стає еталоном, відносно якого оцінюють наступні дані. Тут визначається оптимальна кількість факторів, за якими вестимуться спостереження. їх сукупність не повинна бути надто великою, оскільки це знизить загальну продуктивність роботи. Вона також не повинна бути надто обмеженою, оскільки за недостатньо вичерпними характеристиками неможливо буде побудувати профіль нормальної поведінки. Можливі два загальні види помилок:

а) нормальна поведінка системи або користувача помилково приймається за зловмисну (false positives);

б) спроба зловмисного проникнення в систему приймається за нормальну активність (false negatives).

Хоча жодна з цих ситуацій небажана, але друга все ж таки небезпечніша, і тому одним з основних завдань побудови ADS є чітке визначення умов, за яких ситуація сприймається як аномальна, так, щоб жодна з перелічених ситуацій не виникала занадто часто \cite{axelsson1999base}.

\subsection{Модель Дороті Денінга}
\subsubsection{Принцип роботи}
Система виявлення вторгнення в реальному часі базується на гіпотезі, що 
експлуатація вразливостей системи включає аномальне використання системи.
Тому, порушення безпеки можуть бути виявлені як відмінне від звичайного використання системи
\cite{denning1987intrusion}.

Модель системи заснована на правилах перевірки відповідності шаблону. Коли
створюється запис аудиту, він порівнюється з профілями користувачів, додає
інформацію у відповідні профілі, потім визначає, які правила застосовувати для
оновлення профілів, перевірки аномальної поведінки і як повідомляти, що аномалії
виявлені. Щаблони профілів визначаються оператором з системи моніторингу, але
правила і профілі структури в певній мірі незалежні від системи.
% у меня не так, у меня профили тоже делает система (только отдельный модуль)

Основна ідея полягає в спостереженні за стандартними операціями у цільовій
системі: входи, команди, запуск програм, доступ до файлів і пристроїв і т.д. 
Модель не містить будь-яких особливих механізмів для визначення складних дій, 
які використовують відомі або підозрілі дефекти в безпеці цільової системи; 
дійсно, система не має ніяких знань про механізми безпеки на цільовій системі 
або їх недоліки. Хоча механізм виявлення вторгнень на основі відомих або 
підозрілих дефектів може бути достатьно ефективним, було б значно складніше 
упоратися із вторгненнями, які використовують недоліки, що не підозрюються або с 
пов'язані із персоналом, який обслуговує цільову систему.

\subsubsection{Модель Денінга}

Профілі активності характеризують поведінку суб'єктів, тобто слугує "підпісом", 
або описом нормальної поведінки суб'єктів. Зауважена поведінка характеризується 
в рамках статистичних метрик та моделей. Метрика - випадкова змінна $x$, що 
характеризує дискретну величину - кількість відповідних подій, що пройшли за 
певний період часу. Період може бути фіксований, або між двома іншими подіями
(вхід та вихід з системи). Спостереження (вибірка) дістається із записів аудиту 
системи (журналу) та за допомогою статистичної моделі визначають чи є нова 
спостережена величина аномальною. Статистична модель не робить припущень про
характер розподілення змінної, а визначає його із проведених спостережень.

Розглянуто три типа метрик:

\begin{enumerate}
  \item лічильник подій --- $x$ це кількість записів, що 
        задовільняють деякі умови за певний період часу.
  \item інтервал часу --- $x$ це інтервал часу між двома 
        пов'язаними подіями.
  \item вимір ресурсу --- $x$ це об'єм або кількість ресурсів,
        що задіяні протягом виконання одного запису журналу.
\end{enumerate}

Спосіб визначення аномалій:

\begin{enumerate}
  \item Фіксовані границі змінної $x$
  \item Середнє із середньоквадратичним відхиленням: базується на
        припущенні, що все, що ми знаємо про розподіл
        $x_1, x_2, ..., x_n$ це середнє значення
        \begin{equation}
            \overline{x} = \frac{1}{n} \sum{x_i}
        \end{equation}
        та середньоквадратичне відхилення
        \begin{equation}
            \sigma_x = \sqrt{\frac{1}{n} \sum{(x_i - \overline{x})^2}}
        \end{equation}

        Нове спостереження $x_{n+1}$ є відхиленням, якщо випадає за рамки довірчого інтервалу

        \begin{equation}
            \overline{x} - d \cdot \sigma_x < x_{n+1} < \overline{x} + d \cdot \sigma_x
        \end{equation}

    \item Багатовимірна модель: схожа на попередню модель, але
          використовує кореляцію одразу по декільком змінним.
    \item Модель марковського процесу
    \item Модель на числових рядах: враховує не тільки кількість
          подій, а й проміжки часу між ними.
\end{enumerate}

Дані змінних активності користувачів які використовуються
у профілі, можна агрегувати декільками способами: % 7 страница

\begin{itemize}
  \item класс-як-єдине -- множина всіх суб'єктів та об'єктів
        розглядається як одне ціле, всі події та змінні
        агрегуються по всьому класу.
  \item індивідуальна активність -- суб'єкти або об'єкти в 
        системі розглядаються як окремі сутності. Події та змінні агрегуються по сутностям окремо.
\end{itemize}

\subsubsection{Висновки}
    У роботі розглянута модель системи запобігання вторгненням у цільому систему в реальному часі, розглянуто більшість можливих варіацій компонентів системи та підходів до розв'язку задачі виявлення аномалій у поведінці користувачів.

    Розроблена модель призначена для виявлення вторгнень в
    комп'ютерну систему, а не телефонну.

    Можливе розглядання користувачів окремо, або всю цільову
    систему вцілому, комплексний підхід не врахований.

\subsection{Виброси ? \TBD}
    \TBD \cite{ben2005outlier}.


\newpage
\subsection*{Висновки до розділу 2}
\addcontentsline{toc}{subsection}{Висновки до розділу 2}
    \TBD
