\newpage
\bignumberedpart{Огляд існуючих розв'язків}
\subsection{Модель системи виявлення вторгнень в реальному часі}
\subsubsection{Принцип роботи}
Система виявлення вторгнення в реальному часі базується на гіпотезі, що експлуатація вразливостей системи включає аномальне використання системи. Тому, порушення безпеки можуть бути виявлені як аномальні використання системи \cite{denning1987intrusion}.

Модель системи заснована на правилах перевірки відповідності шаблону. Коли створюється запис аудиту, він порівнюється з профілями користувачів, додає інформацію у відповідні профілі, потім визначає, які правила застосовувати для оновлення профілів, перевірки аномальної поведінки і як повідомляти, що аномалії виявлені. Щаблони профілів визначаються оператором з системи моніторингу, але правила і профілі структури в певній мірі незалежні від системи. % у меня не так, у меня профили тоже делает система (только отдельный модуль)

Основна ідея полягає в спостереженні за стандартними операціями у цільовій системі: входи, команди, запуск програм, доступ до файлів і пристроїв і т.д. Модель не містить будь-яких особливих механізмів для визначення складних дій, які використовують відомі або підозрілі дефекти в безпеці цільової системи; дійсно, система не має ніяких знань про механізми безпеки на цільовій системі або їх недоліки. Хоча механізм виявлення вторгнень на основі відомих або підозрілих дефектів може бути достатьно ефективним, було б значно складніше упоратися із вторгненнями, які використовують недоліки, що не підозрюються або с пов'язані із персоналом, який обслуговує цільову систему.

\subsubsection{Реалізація / мат.модель ?\TBD}

Профілі активності характеризують поведінку суб'єктів, тобто слугує "підпісом", або описом нормальної поведінки суб'єктів. Зауважена поведінка характеризується в рамках статистичних метрик та моделей. Метрика - випадкова змінна $x$, що характеризує дискретну величину - кількість відповідних подій, що пройшли за певний період часу. Період може бути фіксований, або між двома іншими подіями (вхід та вихід з системи). Спостереження (вибірка) дістається із записів аудиту системи (журналу) та за допомогою статистичної моделі визначають чи є нова спостережена величина аномальною.

Спосіб визначення аномалій:

    Дані змінних активності користувачів які використовуються у профілі, можна агрегувати декільками способами: % 7 страница

\begin{itemize}
  \item класс-як-єдине -- множина всіх суб'єктів та об'єктів розглядається як одне ціле
  \item \TBD
\end{itemize}

    У роботі  розглянута модель системи запобігання вторгненням в реальному часі.

	Мінуси:

	1. Розроблена модель призначена для виявлення вторгнень в комп'ютерну систему, а не телефонну.

	2. Система не контролює профілі користувачів. ???? % p.4

	3. Можливе розглядання користувачів окремо, або всю цільову систему вцілому, комплексний підхід не врахований. \TBD

\subsection{Виброси ? \TBD}
    \TBD \cite{ben2005outlier}.


\newpage
\subsection*{Висновки}
\addcontentsline{toc}{subsection}{Висновки}
    \TBD
