% \section*{Вступ}
%notoc \addcontentsline{toc}{section}{Вступ}
\newpage
\bigpart{Вступ}
	
	Проблема раннього виявлення шахрайства наразі стоїть дуже гостро, оператори зв'язку по всьому світу відчувають значні втрати через шахраїв. За даними дослідження CFCA (міжнародна організація для контролю шахрайства, забезпечення доходів і запобігання втрат) в 2013 році, втрати становлять 46.3 мільярдів доларів на рік, що більше на 15\% порівняно з аналогічним дослідженням в 2011 році. Із збільшенням втрат, 8\% компаній переклали функції з боротьби з шахрайством з фінансових відділів у відділи ІТ і безпеки (зараз 38\% від усіх компаній) \cite{cfca2013survey}.

	Ігнорувати проблему шахрайства неможливо, так само як і припинити, тому мета - виявити шкідливе втручання в канали і засоби зв'язку, неправомірне використання послуг якомога раніше і запобігти його.

	Системи моніторингу трафіку визначають аномальний трафік, використовуючи аналіз і побудову шаблону поведінки окремого користувача \cite{telenik2009detection} \cite{rosastelecommunications}. Такі системи зазвичай є частиною системи безпеки і не є самодостатніми - при виявленні інтервенції, залежно від політики безпеки, абонент відразу блокується або подається сигнал співробітнику оператора для ручної обробки.

	Проблемою такого підходу є періодичні і єдиноразові масові зміни поведінки абонентів системи. Прикладом можуть служити свята, як Новий рік, соціально-політичні події і багато іншого. При цьому такі події по- різному впливають на користувачів з різними шаблонами поведінки, наприклад корпоративних користувачів навряд чи торкнуться сімейні свята, а такі дні як "чорна п'ятниця" вплинуть на кількість вихідних дзвінків корпоративного сегмента, але не приватних користувачів \cite{rader2014cdr}.

	Дана робота присвячена створенню системи раннього виявлення шахрайства із урахуванням вказаної проблеми, для цього користувачі будуть кластеризовані у групи, всередині яких буде проводитися пошук аномалій.
