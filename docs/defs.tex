% vim: nocindent:

\hyphenation
{
  три-ва-лість
  Від-по-відь
  Open-FOAM
  ал-го-рит-мів
  ал-го-рит-мом
  ал-го-рит-му
  апро-кси-ма-ція-ми
  біб-ліо-те-ка
  век-тор
  век-то-ри
  век-то-рів
  виг-ля-ді
  від-нос-но
  від-по-ві-да-ють
  влас-ний
  влас-них
  вуз-ла
  вхід-них
  дея-кі
  до-дат-ко-ві
  ефек-тив-ним
  зас-то-со-вує-ться
  зруч-но-му
  ка-фед-рою
  квад-рат
  квад-рат-ної
  ква-лі-фі-ка-цій-но
  клас-тер-ної
  кое-фі-ці-єнт
  мак-си-маль-ної
  мат-ри-ці
  мат-риць
  мат-ри-ця
  мож-ли-вим
  нас-туп-ні
  не-мож-ли-во
  об-роб-ля-ти
  об-чис-ленні
  об-чис-лення
  об-чис-лень
  об-чис-лю-валь-ни-ми
  об-чис-лю-валь-но
  об-чис-лю-валь-ної
  пе-ре-тво-рен-ня
  пос-лі-дов-нос-ті
  прак-тич-не
  пре-фікс-ної
  про-дук-тив-ність
  різ-но-ма-ніт-них
  роз-рід-же-ної
  си-ме-трич-них
  си-ме-трич-ної
  сис-тем
  сис-те-мі
  склад-ність
  склад-но-сті
  цик-лу
}

\ESKDdepartment{Міністерство освіти і науки України}
\ESKDcompany{Національний технічний університет України%
<<Київський політехнічний інститут>>}

\ESKDauthor{Радер~Р.\,І.}
\ESKDchecker{Кулаков~Ю.\,О.}
\ESKDnormContr{Симоненко~В.\,П.}
\ESKDapprovedBy{Луцький~Г.\,М.}
\ESKDdate{2014/01/16}
\ESKDcolumnIX{НТУУ <<КПІ>> ФІОТ ІО-02}
\ESKDgroup{НТУУ <<КПI>> \\ група IО-02}

\ESKDsignature{IАЛЦ 467531.002 ПЗ}
% \ESKDauthor{Радер Р.І.}
% \ESKDchecker{Кулаков Ю.О.}
\ESKDtitle{\small{Система виявлення аномальної поведінки абонентів телефонної мережі}}
\ESKDdocName{Пояснювальна записка}


% Set text padding
\ESKDsetPadding{10mm}{10mm}

% Redefine TOC styles
\makeatletter
\renewcommand{\l@section}{\@dottedtocline{0}{1.5em}{2.3em}}
\renewcommand{\l@subsection}{\@dottedtocline{1}{2.5em}{2.3em}}
\renewcommand{\l@subsubsection}{\@dottedtocline{2}{3.5em}{2.3em}}
\makeatother

% Document separator
\newcommand{\docseparator}[1]{
  \newpage
  \thispagestyle{empty}
  \ESKDthisStyle{empty}
  \noindent\parbox[c][\vsize][c]{\hsize}
  {\centering\fontsize{36pt}{40pt}\selectfont#1}

  % reset all counters
  \setcounter{page}{0}
  \setcounter{section}{0}
  \setcounter{subsection}{0}
  \setcounter{subsubsection}{0}
}

\newcommand{\apptitletop}[1]{%
ДОДАТОК #1\par
Алгоритм динамічної маршрутизації мультікастової розсилки}

\newcommand{\apptitlepages}[1]{%
Аркушів \pageref{LastPage}\par}

% Appendix title page
\newcommand{\appendixtitle}[3]{
  \newpage
  \thispagestyle{empty}
  \ESKDthisStyle{empty}
  \noindent
  \parbox[t][0.15\vsize][t]{\hsize}
  {\centering\large\apptitletop{#1}}
  \parbox[t][0.4\vsize][c]{\hsize}
  {\centering\Large\textbf{#2}\par#3}
  \parbox[t][0.20\vsize][c]{\hsize}{~~}
  \parbox[t][0.08\vsize][t]{\hsize}
  {\centering\apptitlepages}
  \vfill
  \noindent\parbox[c]{\hsize}{\centering\titlebottom}

  % reset all counters
  \setcounter{page}{0}
  \setcounter{section}{0}
  \setcounter{subsection}{0}
  \setcounter{subsubsection}{0}
}

% Header
\newcommand{\titletop}{%
\textbf{МІНІСТЕРСТВО ОСВІТИ І НАУКИ УКРАЇНИ}\par
\textbf{НАЦІОНАЛЬНИЙ ТЕХНІЧНИЙ УНІВЕРСИТЕТ УКРАЇНИ}\par
\textbf{<<КИЇВСЬКИЙ ПОЛІТЕХНІЧНИЙ ІНСТИТУТ>>}\par
Факультет інформатики і обчислювальної техніки\par
Кафедра обчислювальної техніки}

\newcommand{\titledocname}{%
БАКАЛАВРСЬКА ДИПЛОМНА РОБОТА\par
НА ТЕМУ}

\newcommand{\titledocnameI}{%
БАКАЛАВРСЬКА ДИПЛОМНА РОБОТА}

\newcommand{\titledocnameII}{%
БАКАЛАВРСЬКА ДИПЛОМНА РОБОТА\par
НА ТЕМУ}

\newcommand{\titleregistry}{%
\textbf{Реєстраційний \No} \underline{~~~~~~~~~~}}

\newcommand{\titlescript}{%
\textbf{На правах рукопису}\par
УДК \underline{~~~~~~~~~~~~~~~~~~~~}}

\newcommand{\titleapproveI}{%
\textbf{Затверджую}\par
зав. кафедрою, д.т.н., проф.\par
\underline{~~~~~~~~~~~~~~~} (Луцький~Г.\,М.)\par
\vspace{-3mm}{\small(підпис, дата)}}

\newcommand{\titleapproveII}{%
\textbf{Узгоджено}\par
Науковий керівник\par
д.т.н., проф.\par
Кулаков Юрій Олексiйович\par
~\par
\underline{~~~~~~~~~~~~~~~~~~~~~~~~~~~~~}\par
\vspace{-3mm}{\small~~~~~~~~(підпис, дата)}}

\newcommand{\titledefence}{%
Захищено <<\underline{~~~}>> \underline{~~~~~~~~~~} 2014 р.\par
З оцінкою \underline{~~~~~~~~~~~~~~~}\par
Члени комісії\par
\underline{~~~~~~~~~~~~~~~~~~~~~~~~~~~~~}\par
\vspace{2mm}\underline{~~~~~~~~~~~~~~~~~~~~~~~~~~~~~}\par
\vspace{2mm}\underline{~~~~~~~~~~~~~~~~~~~~~~~~~~~~~}}

\newcommand{\titletheme}{%
\textbf{\underline{Система виявлення аномальної поведінки}}\par
\textbf{\underline{абонентів телефонної мережі}}}

\newcommand{\titlesubtheme}{%
Зі спеціальності (за напрямком) 6.050102\par
<<Комп'ютерна інженерія>>}

\newcommand{\titleauthorI}{%
\textbf{Виконавець роботи}\par
ФІОТ, гр. ІО-02\par
номер залікової книжки: 0215\par
Радер Роман Ілліч\par
\vspace{2mm}\underline{~~~~~~~~~~~~~~~~~~~~~~~~~~~~~}\par
\vspace{-3mm}{\small~~~~~~~~(підпис, дата)}}

\newcommand{\titleauthorII}{%
\textbf{Виконавець роботи}\par
Радер Роман Ілліч\par
\vspace{2mm}\underline{~~~~~~~~~~~~~~~~~~~~~~~~~~~~~}\par
\vspace{-3mm}{\small~~~~~~~~(підпис, дата)}}

\newcommand{\titleadviser}{%
\textbf{Науковий керівник}\par
к.т.н., доц.\par
Кулаков Юрій Олексiйович\par
~\par
\vspace{2mm}\underline{~~~~~~~~~~~~~~~~~~~~~~~~~~~~~}\par
\vspace{-3mm}{\small~~~~~~~~(підпис, дата)}}

\newcommand{\titlebottom}{%
Київ 2014~р.}

\renewcommand{\maketitle}{
  \thispagestyle{empty}
  \ESKDthisStyle{title}

  \noindent\parbox[c][0.15\vsize][t]{\hsize}
  {\begin{center}\titletop\end{center}}
  \parbox[t][0.08\vsize][t]{\hsize}
  {
    \parbox[t]{0.4\hsize}{\raggedright\titleregistry}\hfill
    \parbox[t]{0.4\hsize}{\raggedright\titlescript}
  }
  \parbox[t][0.1\vsize][t]{\hsize}
  {
    \parbox[t]{0.4\hsize}{\raggedright}\hfill
    \parbox[t]{0.4\hsize}{\raggedright\titleapproveI}
  }
  \parbox[t][0.05\vsize][c]{\hsize}{~~}
  \parbox[t][0.08\vsize][c]{\hsize}
  {\begin{center}\titledocname\end{center}}
  \parbox[t][0.08\vsize][c]{\hsize}
  {\begin{center}\titletheme\end{center}}
  \parbox[t][0.08\vsize][c]{\hsize}
  {\begin{center}\titlesubtheme\end{center}}
  \parbox[t][0.08\vsize][c]{\hsize}{~~}
  \parbox[t][0.2\vsize][t]{\hsize}
  {
    \parbox[t]{0.45\hsize}{\raggedright\titleauthorI}\hfill
    \parbox[t]{0.4\hsize}{\raggedright\titleadviser}
  }
  \vfill
  \begin{center}\titlebottom\end{center}
}

\newcommand{\maketitleI}{
  \thispagestyle{empty}
  \ESKDthisStyle{title}

  \noindent\parbox[c][0.20\vsize][t]{\hsize}
  {\centering\titletop}
  \parbox[t][0.07\vsize][t]{\hsize}
  {
    \parbox[t]{0.4\hsize}{\raggedright\titleregistry}\hfill
    \parbox[t]{0.4\hsize}{\raggedright\titlescript}
  }
  \parbox[t][0.15\vsize][t]{\hsize}
  {
    \parbox[t]{0.4\hsize}{\raggedright}\hfill
    \parbox[t]{0.4\hsize}{\raggedright\titleapproveI}
  }
  \parbox[t][0.08\vsize][c]{\hsize}
  {\centering\titledocnameI}
  \parbox[t][0.08\vsize][c]{\hsize}
  {\centering\titlesubtheme}
  \parbox[t][0.12\vsize][c]{\hsize}{~~}
  \parbox[t][0.2\vsize][t]{\hsize}
  {
    \parbox[t]{0.45\hsize}{\raggedright\titleauthorI}\hfill
    \parbox[t]{0.4\hsize}{\raggedright\titleadviser}
  }
  \vfill
  \noindent\parbox[c]{\hsize}{\centering\titlebottom}
}

\newcommand{\maketitleII}{
  \thispagestyle{empty}
  \ESKDthisStyle{title}

  \noindent\parbox[c][0.2\vsize][t]{\hsize}
  {\centering\titletop}
  \parbox[t][0.20\vsize][t]{\hsize}
  {
    \parbox[t]{0.45\hsize}{\raggedright\titleapproveII}\hfill
    \parbox[t]{0.45\hsize}{\raggedright\titledefence}
  }
  \parbox[t][0.05\vsize][c]{\hsize}{~~}
  \parbox[t][0.08\vsize][c]{\hsize}
  {\centering\titledocnameII}
  \parbox[t][0.08\vsize][c]{\hsize}
  {\centering\titletheme}
  \parbox[t][0.08\vsize][c]{\hsize}
  {\centering\titlesubtheme}
  \parbox[t][0.10\vsize][c]{\hsize}{~~}
  \parbox[t][0.15\vsize][t]{\hsize}
  {
    \parbox[t]{0.45\hsize}{\raggedright}\hfill
    \parbox[t]{0.45\hsize}{\raggedright\titleauthorII}
  }
  \vfill
  \noindent\parbox[c]{\hsize}{\centering\titlebottom}
}

\makeatletter
\newcommand\bigpart[1]{%
  \begingroup%
  \ESKDsectAlign{section}{Center}%
  \section*{#1%
    \@mkboth{%
      \MakeUppercase#1}{\MakeUppercase#1}}%
  \addcontentsline{toc}{section}{#1}%
  \endgroup}
\newcommand\bignotocpart[1]{%
  \begingroup%
  \ESKDsectAlign{section}{Center}%
  \section*{#1%
    \@mkboth{%
      \MakeUppercase#1}{\MakeUppercase#1}}%
  \endgroup}
\newcommand\bignumberedpart[1]{%
  \begingroup%
  \ESKDsectAlign{section}{Center}%
  \stepcounter{section}%
  \section*{Розділ \arabic{section}}%
  \vspace{-5mm}%
  \section*{#1%
    \@mkboth{%
      \MakeUppercase#1}{\MakeUppercase#1}}%
  \addcontentsline{toc}{section}{Розділ \arabic{section}. #1}%
  \endgroup}
\makeatother

\newcommand{\vect}[1]{\mathbf{#1}}
\DeclareMathOperator{\const}{const}
\DeclareMathOperator{\proj}{\mathbf{proj}}
\DeclareMathOperator{\rang}{\mathbf{rang}}
\DeclareMathOperator{\opspan}{\mathbf{span}}
\DeclareMathOperator{\opdim}{\mathbf{dim}}
\DeclareMathOperator*{\argmin}{arg\,min}

% theorem-like environments
\theoremstyle{plain}
\theorembodyfont{}
\theoremseparator{.}
\newtheorem{example}{Приклад}
\newtheorem{theorem}{Теорема}

\theoremstyle{nonumberplain}
\theoremseparator{.}
\theoremsymbol{\rule{1ex}{1ex}}
\newtheorem{proof}{Доведення}

\numberwithin{equation}{section}
\numberwithin{theorem}{section}
\numberwithin{figure}{section}
\numberwithin{table}{section}

\makeatletter
% \def\verbatim@font{\fontfamily{fcr}\fontsize{10pt}{11pt}\selectfont\@noligs}
\makeatother

% remove some hyperref warnings
\makeatletter
\providecommand*{\toclevel@theorem}{0}
\providecommand*{\toclevel@proof}{0}
\makeatother

\newcommand\algorule{\rule{\textwidth}{0.5pt}}

% inline program code
\newcommand{\icode}[1]{\texttt{\small#1}}

% zero-width space (allow breaking lines)
\newcommand{\zwsp}{\hspace{0mm}}


% to be done block
\usepackage{color}
\newcommand{\TBD}{{\color{red}To be done}}



\SetKwInput{KwData}{Вхідні параметри}
\SetKwInput{KwResult}{Результат}
\SetKwInput{KwIn}{Вхідні дані}
\SetKwInput{KwOut}{Вихідні данные}
\SetKwIF{If}{ElseIf}{Else}{якщо}{тоді}{інакше\ якщо}{інакше}{кінець\ умови}
\SetKwFor{While}{до\ тих\ пір,\ поки}{виконувати}{кінець\ циклу}
\SetKw{KwTo}{до}
\SetKw{KwRet}{повернути}
\SetKw{Return}{повернути}
\SetKwBlock{Begin}{початок\ блоку}{кінець\ блоку}
\SetKwSwitch{Switch}{Case}{Other}{Перевірити\ значення}{та\ виконати}{варіант}{інакше}{кінець\ варианту}{кінець\ перевірки\ значень}
\SetKwFor{For}{цикл}{виконувати}{кінець\ циклу}
\SetKwFor{ForEach}{для\ кожного}{виконувати}{кінець\ циклу}
\SetKwRepeat{Repeat}{повторювати}{до\ тих\ пір,\ поки}
\SetAlgorithmName{Алгоритм}{алгоритм}{Список алгоритмів}
